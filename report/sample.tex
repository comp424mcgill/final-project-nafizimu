\documentclass[twoside,11pt]{article}

% Any additional packages needed should be included after jmlr2e.
% Note that jmlr2e.sty includes epsfig, amssymb, natbib and graphicx,
% and defines many common macros, such as 'proof' and 'example'.
%
% It also sets the bibliographystyle to plainnat; for more information on
% natbib citation styles, see the natbib documentation, a copy of which
% is archived at http://www.jmlr.org/format/natbib.pdf

\usepackage{jmlr2e}
% Definitions of handy macros can go here
\usepackage{subfig}
\newcommand{\dataset}{{\cal D}}
\newcommand{\fracpartial}[2]{\frac{\partial #1}{\partial  #2}}

% Heading arguments are {volume}{year}{pages}{submitted}{published}{author-full-names}

% \jmlrheading{1}{2000}{1-48}{4/00}{10/00}{Marina Meil\u{a} and Michael I. Jordan}

% Short headings should be running head and authors last names

% \firstpageno{1}

\begin{document}
\title{COMP 424 Final Project Game: \textit{Colosseum Survival!}}
\ShortHeadings{Final project COMP 424, McGill University}{}
\author{Team Name: NafiZimu\\\\Team Members: \\Zimu Su (\texttt{zimu.su@mail.mcgill.ca}) \\Nafiz Islam (\texttt{nafiz.islam@mail.mcgill.ca})}

\maketitle

\section{Introduction}
% how program works and what are the motivation of the approach

\section{Theory}

\section{Analysis}

\section{Other Approach}

\section{Possible Improvement}

\section{Report}

You are required to write a report with a detailed explanation of your approach and reasoning. The report must be a typed PDF file, and should be free of spelling and grammar errors. The suggested length is between 4 and 8 pages, but the most important constraint is that the report be clear and concise. You should use the source of this document \footnote{You can find the source of this document in Overleaf here: \href{https://www.overleaf.com/read/gcpfjdpqpytp}{https://www.overleaf.com/read/gcpfjdpqpytp}. You would need to create an Overleaf account. Copy this project and create a new project to write your report. Replace the \texttt{author} information with your own group information.} as a template to write your report in \LaTeX.  The report must include the following required components:

\begin{itemize}
    \item An explanation of how your program works, and a motivation for your approach.
    \item A brief description of the theoretical basis of the approach (about a half-page in most cases); references to the text of other documents, such as the textbook, are appropriate but not absolutely necessary. If you use algorithms from other sources, briefly describe the algorithm and be sure to cite your source.
    \item A summary of the advantages and disadvantages of your approach, expected failure modes, or weaknesses of your program.
    \item If you tried other approaches during the course of the project, summarize them briefly and discuss how they compared to your final approach.
    \item A brief description (max. half page) of how you would go about improving your player (e.g. by introducing other AI techniques, changing internal representation etc.)
    
\end{itemize}

% \vskip 0.2in
% \bibliography{sample}

\end{document}