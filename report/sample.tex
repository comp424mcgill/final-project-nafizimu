\documentclass[twoside,11pt]{article}

% Any additional packages needed should be included after jmlr2e.
% Note that jmlr2e.sty includes epsfig, amssymb, natbib and graphicx,
% and defines many common macros, such as 'proof' and 'example'.
%
% It also sets the bibliographystyle to plainnat; for more information on
% natbib citation styles, see the natbib documentation, a copy of which
% is archived at http://www.jmlr.org/format/natbib.pdf

\usepackage{jmlr2e}
% Definitions of handy macros can go here
\usepackage{subfig}
\usepackage[skip=1em, indent=0pt]{parskip}
\newcommand{\dataset}{{\cal D}}
\newcommand{\fracpartial}[2]{\frac{\partial #1}{\partial  #2}}

% Heading arguments are {volume}{year}{pages}{submitted}{published}{author-full-names}

% \jmlrheading{1}{2000}{1-48}{4/00}{10/00}{Marina Meil\u{a} and Michael I. Jordan}

% Short headings should be running head and authors last names

% \firstpageno{1}

\begin{document}
\title{COMP 424 Final Project Game: \textit{Colosseum Survival!}}
\ShortHeadings{Final project COMP 424, McGill University}{}
\author{Team Name: NafiZimu\\\\Team Members: \\Zimu Su (\texttt{zimu.su@mail.mcgill.ca}) \\Nafiz Islam (\texttt{nafiz.islam@mail.mcgill.ca})}

\maketitle

\section{Introduction}

Our agent is a Python module named \texttt{student\_agent.py} that consists of classes called \texttt{MCTSNode} and \texttt{StudentAgent}.

\subsection{\texttt{StudentAgent}}

\texttt{StudentAgent} is the entry point of the module for the \textit{Colosseum Survival!} game.
It consists of the \texttt{\_\_init\_\_} method to construct the agent, helper functions, global variables, and a \texttt{step} method.

\texttt{\_\_init\_\_} initializes variables including \texttt{self.is\_first\_round} which is \texttt{True} if the current turn is the first turn.

The \texttt{step} method is one of the public interface for the \textit{Colosseum Survival!} game to make a decision for a single turn.
The method sets \texttt{StudentAgent.end\_time} to the time when the MCTS should stop searching, calls \texttt{StudentAgent.mcts}, and then returns the most optimal next position and direction discovered.

The \texttt{StudentAgent.bfs} is a static method that performs Breadth-First Search.
It returns a generator that returns a sequence of \texttt{(number of steps, point)} where the points are legally accessible from the starting point within the maximum step.
If given the adversary's position, it will not visit the adversary's position.
This function is used to find all the legally accessible points for the player at the starting point.

The \texttt{StudentAgent.greedy\_search} method is a static method that performs a greedy search.
It is greedy because it uses min-heap to prioritize points that are closest to the adversary's point in terms of Manhattan distance.
It also cannot pass through walls, but it ignores the maximum step.
This function serves two purpose: to determine whether both players are on the same region, or to return all the points that are within the same region as the starting point.
The purpose of making the function greedy is to stop calculating the score of a chessboard as soon as it is discovered that both players are still in the same region.

The \texttt{StudentAgent.game\_score} method is a static method that calculates the score of the chessboard.
It returns a \texttt{tuple} where the first value is score of the first player, and the second value is the score of the second player;
otherwise it returns \texttt{None} if both players are on the same region.
The method uses \texttt{greedy\_search} to not only explore the region of the players and calculate their score, but also to determine if both players are on the same region much sooner.

The \texttt{StudentAgent.set\_wall} method is a static method to set the value of a wall in a chessboard.
The method serves two purpose: to conveniently set a wall at opposite sides, and to return \texttt{True} if the wall placed is connected to other walls on both ends of the wall.
The purpose of the latter is to reduce the number of calls to \texttt{StudentAgent.game\_score} in a Monte-Carlo method;
the assumption is that if both ends are not connected, then no new region is created, so the score would be the exact same as previous turn.

The \texttt{StudentAgent.mcts} method is a static method that performs Monte-Carlo Tree Search using \texttt{MCTSNode} until the time limit.
After the time limit is reached, it will return the most optimal next position and direction discovered.

\subsection{\texttt{MCTSNode}}

\texttt{MCTSNode} represents a node in a Monte-Carlo Tree Search.
It could either be the root of the tree, or the children.
It consists of methods such as \texttt{\_\_init\_\_}, \texttt{tree\_policy}, \texttt{back\_propagation}, \texttt{best\_child} and \texttt{monte\_carlo\_method}.

\texttt{\_\_init\_\_} not only initializes fields but also performs default policy.
The purpose of performing default policy is to determine if the node can no longer be expanded.
This allows the tree policy to expand other nodes that are not finished.

\texttt{tree\_policy} performs the tree policy of MCTS to update the score of the nodes along the most promising nodes.

\texttt{back\_propagation} performs the back propagation to update the score of predecessors during a default policy of MCTS.

\texttt{best\_child} returns the best direct child of an \texttt{MCTSNode}.
If the \texttt{greedy} argument is \texttt{False}, then it will pick the child with the highest win/round ratio;
otherwise, it will perform Upper Confidence Tree with a scaling constant of \(\sqrt{2}\) to maintain balance between exploration and exploitation.
\texttt{greedy} is \texttt{True} when performing default policy, and \texttt{False} when picking the best next position and direction at the end of MCTS.

\texttt{monte\_carlo\_method} performs a single Monte-Carlo method and returns the score of the simulation.

There have been some consistent strategies we have applied to minimize overheads of simulating multiple games with default policy, \texttt{tree\_policy} and \texttt{monte\_carlo\_method}.
For example, we would modify the original copy of the chessboard and then undo the changes to avoid overhead related to copying large data.
We also perform iteration instead of recursion to avoid stack overflow and some overheads related to managing call stack.

\section{Theory}
The basic idea we used of our implementation is the Monte Carlo Tree Search (MCTS) method.
By using Monte Carlo Method (MCM), our agent could use the results from random simulations to quickly obtain a reasonable estimation of the win rate of each of its options.
Then, by combining these estimations and creating an Upper Confidence Tree (UCT), our agent can quickly grasp an idea of which option should it take for its next move. 
As the game progress, the number of options our agent have reduces, which makes MCTS able to create a taller UCT. 
Since the deeper the UCT, the more simulations the agent gets to do, the agent's estination becomes more accurate as the height of UCT grows.

Because of the potential high branching factor when the board gets large (8-10), we incorporated more randomness to the MCTS we learned in class. 
Instead of going through all the possible options each time, our agent only explore on roughly half of the options. 
By doing so, the agent could get to a deeper depth with exploitation, which makes its estimation less of a product of luck and more of a reliable estimation.

\section{Analysis}
As mentioned in the theory section, our agent mainly follows MCTS method. 
We made some changes to the normal MCTS so it favors exploitation over exploration. 
And, we gave it a internal "time limit" for it so it always finishes its decision within the allowed time.

\subsection{Advantages}
One of the main advantages of our agent is that it is unlikely to take more than the allowed time to select a move. 
In other words, we configured our agent so that it has a low "misplay rate". 
In order to do so we have to sacrifice some of the exploration options. This means that our agent might miss some really good opportunities to close the game off. 
But mroe generally speaking, this could make our agent always walk on a path that favors our win a little bit more, and we believe this trade off between risk and consistency is worth it.

\subsection{Disadvantages}
\subsection{Failure Modes}

\section{Other Approach}

\section{Possible Improvement}

There are a few possible improvements that can be made in the Monte-Carlo Tree Search.
The simplest improvement that can be made is to prioritize adjacent state spaces that would add a wall that is connected to another wall.
This would maximize the value of the children nodes since a human player would intuitively place a wall that is connected to another wall to quickly create a closed region.
Another possible optimization is to reuse an existing Monte-Carlo tree search.
This would allow us to take advantage of the 30 seconds in the first round to build a large search tree and then reuse a subsection of the tree with some probability to avoid rebuilding an entire tree for the subsequent rounds.

\section{Report}

You are required to write a report with a detailed explanation of your approach and reasoning. The report must be a typed PDF file, and should be free of spelling and grammar errors. The suggested length is between 4 and 8 pages, but the most important constraint is that the report be clear and concise. You should use the source of this document \footnote{You can find the source of this document in Overleaf here: \href{https://www.overleaf.com/read/gcpfjdpqpytp}{https://www.overleaf.com/read/gcpfjdpqpytp}. You would need to create an Overleaf account. Copy this project and create a new project to write your report. Replace the \texttt{author} information with your own group information.} as a template to write your report in \LaTeX.  The report must include the following required components:

\begin{itemize}
    \item An explanation of how your program works, and a motivation for your approach.
    \item A brief description of the theoretical basis of the approach (about a half-page in most cases); references to the text of other documents, such as the textbook, are appropriate but not absolutely necessary. If you use algorithms from other sources, briefly describe the algorithm and be sure to cite your source.
    \item A summary of the advantages and disadvantages of your approach, expected failure modes, or weaknesses of your program.
    \item If you tried other approaches during the course of the project, summarize them briefly and discuss how they compared to your final approach.
    \item A brief description (max. half page) of how you would go about improving your player (e.g. by introducing other AI techniques, changing internal representation etc.)
    
\end{itemize}

% \vskip 0.2in
% \bibliography{sample}

\end{document}